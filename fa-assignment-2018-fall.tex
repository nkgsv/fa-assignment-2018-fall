\documentclass{article}
\usepackage[utf8]{inputenc}
\usepackage[T2A]{fontenc}
\usepackage[unicode]{hyperref}
\usepackage{xcolor}
\usepackage[english,russian]{babel}
% General document formatting
\usepackage[margin=0.7in]{geometry}
\usepackage[parfill]{parskip}
\usepackage{multicol}
\hypersetup{
  colorlinks=true,
  linkcolor=gray, 
  urlcolor=blue           % color of external links
}

% Related to math
\usepackage{amsmath,amssymb,amsfonts,amsthm,mathrsfs}
\theoremstyle{definition}
\newtheorem{problem}{Задача}
\newenvironment{solution}{\begin{proof}[Решение]}{\end{proof}}
\newcommand{\R}{\mathbb R}
\newcommand{\Q}{\mathbb Q}
\newcommand{\N}{\mathbb N}
\newcommand{\eps}{\varepsilon}
\newcommand{\Ker}{\mathop{\mathrm{Ker}}}

\begin{document}

\title{Задания по функциональному анализу}
\date{\href{https://mipt.ru/education/chair/mathematics/process/progs1/3_kyrs/10\%D1\%84.pdf}{Осенний семестр 2018-2019}}
\author{ФУПМ}
\maketitle

\tableofcontents

\section{Первое задание (срок сдачи 15-19 октября)}


\subsection{Частично упорядоченные множества. Лемма Цорна}


\begin{problem}[T-1.1]
Пусть $X$ --- линейное пространство, $L\subset X$ --- подпространство.
Докажите, что существует подпространство $M\subset X$ такое, что
\begin{equation*}
L+M = X \quad\text{и}\quad L \cap M = \emptyset.
\end{equation*}
\end{problem}

\begin{problem}[Т-1.2]
Пусть $X$ --- нетривиальное линейное пространство.
Множество $\Gamma \subset X$ называется \emph{базисом Г\'{а}меля},
если любой вектор $x\in X$ можно единственным образом представить
в виде конечной линейной комбинации элементов $\Gamma$.
Докажите, что в $X$ существует базис Гамеля.
\end{problem}

\begin{problem}[Т-1.3]
Пусть $X$ и $Y$ --- линейные пространства, $A\colon X\to Y$ ---
сюръективное линейное отображение. Докажите, что существует
линейное отображение $T\colon Y\to X$ такое, что $ATy = y$
для всех $y\in Y$.
\end{problem}

\subsection{Топологические пространства и их компактные подмножества}

\begin{problem}[Т-1.4]
Пусть $F$ --- множество всех функций $f\colon [0,1]\to [0,1]$
с топологией $\tau$ поточечной сходимости.
Пусть $\tau_0$ --- топология в $\R$ с базой из всевозможных
интервалов.
\begin{enumerate}
\item Являются ли отображения $I\colon (F,\tau) \to (\R, \tau_0)$
и $S\colon (F,\tau) \to (\R, \tau_0)$, заданные равенствами
\begin{equation*}
I(f) = \int_0^1 f(x) \, dx,
\quad
S(f) = \sum_{k=1}^\infty 2^{-k} f(2^{-k}),
\quad
f\in F,
\end{equation*}
\begin{multicols}{2}
\begin{itemize}
\item[а)] топологически непрерывными;
\item[б)] секвенциально непрерывными?
\end{itemize}
\end{multicols}
\item Докажите, что $(F, \tau)$ --- компактное топологическое пространство,
не являющееся секвенциально компактным.
\item Приведите пример множества $M\subset F$, которое является секвенциальным компактом и не является топологическим компактом в $(F,\tau)$.
\end{enumerate}
\end{problem}

\begin{problem}[Т-1.5]
Множество $C[0,1]$ состоит из всех непрерывных на отрезке $[0,1]$ функций
$f\colon [0,1] \to \R$.
\begin{enumerate}
\item Докажите, что семейство множеств
\begin{equation*}
\beta = \left\{V_\eps(f) = \left\{g\in C[0,1] \;:\; \left|\int_0^1(f(x) - g(x))\,dx\right| < \eps \right\} \;\middle|\; f\in C[0,1], \eps>0 \right\}
\end{equation*}
образует базу некоторой топологии $\tau$ в $C[0,1]$.
\item Для любой функции $f\in C[0,1]$ найдите замыкание одноточечного множества $\{f\}\subset C[0,1]$ в топологическом пространстве $(C[0,1], \tau)$.
\end{enumerate}
\end{problem}

\begin{problem}[Т-1.6]
Пусть $F$ --- множество всех функций $f\colon \R\to\R$ с топологией $\tau$
поточечной сходимости. Докажите, что $(F,\tau)$ --- топологическое векторное
пространство относительно поточечных операций сложения функций и умножения на вещественные скаляры. Найдите все линейные непрерывные отображения $\Phi\colon (F,\tau) \to \R$.
\end{problem}

\begin{problem}[Т-1.7]
Пусть $F$ --- множество всех функций $f\colon [0,1]\to \R$ с топологией $\tau$ поточечной сходимости.
\begin{enumerate}
\item[а)] Приведите пример множества $S_0 \subset F$, секвенциальное замыкание которого не совпадает с его $\tau$-замыканием в пространстве $(F,\tau)$.
\item[б)] Приведите пример множества $S\subset F$, секвенциальное замыкание которого в пространстве $(F,\tau)$ не является секвенциально замкнутым.
\end{enumerate}
\end{problem}

\begin{problem}[Т-1.8]
Докажите, что топологическое пространство $(X,\tau)$ является компактным
если и только если любое его топологически замкнутое собственное подмножество
является компактным.
\end{problem}

\begin{problem}[Т-1.9]
Докажите, что семейство множеств
\begin{equation*}
\tau = \{V \subset \R : V = \emptyset \text{ или } \R \setminus V \text { -- конечно или пусто}\}
\end{equation*}
является топологией в $\R$.
\begin{enumerate}
\item[а)] Приведите пример компактного и незамкнутого в топологическом пространстве $(\R, \tau)$ множества $K \subset \R$.
\item[б)] Приведите пример секвенциально компактного и секвенциально незамкнутого в топологическом пространстве $(\R, \tau)$ множества $S \subset \R$.
\end{enumerate}
\end{problem}

\subsection{Метрические и линейные нормированные пространства: полнота, сепарабельность, пополнение}

\begin{problem}[\S 1.3]
Является ли открытым в пространстве $C[a,b]$ множество
\begin{equation*}
\{f \in C[a,b] : 0 < f(x) < 1 \quad \forall x \in [a,b]\}?
\end{equation*}
\end{problem}

\begin{problem}[\S 1.4]
Является ли открытым в пространстве $\ell_\infty$ множество
\begin{equation*}
\{x\in \ell_\infty : 0< x_k < 1, \; k=1,2,\ldots\}?
\end{equation*}
(Здесь $x=(x_1,x_2, \ldots)$.)
\end{problem}

\begin{problem}[\S 1.5]
Пусть $A$ --- подмножество метрического пространства $(X,\rho)$.
Докажите, что функция $f\colon X\to \R$, $f(x)=\rho(x,A) = \inf_{y\in A} \rho(x,y)$
непрерывна.
\end{problem}

\begin{problem}[\S 1.7]
Пусть $A, B$ --- замкнутые, непересекающиеся подмножества метрического пространства $X$.
Докажите, что на $X$ существует непрерывная функция $f$ такая, что $f|_A \equiv 0$, $f|_B \equiv 1$.
\end{problem}

\begin{problem}[\S 1.13]
Докажите, что пространство основных функций $\mathscr{D}(\R)$ неметризуемо.
\end{problem}

\begin{problem}[\S 2.2]
Докажите, что пространства $\ell_p$ ($1\le p < \infty$) --- сепарабельные полные метрические пространства,
а пространство $\ell_\infty$ --- полное, но не сепарабельное.
\end{problem}

\begin{problem}[\S 2.3]
Докажите, что если в пространстве $C[a,b]$ рассмотреть метрику $\rho_1(f,g)=\int_a^b |f(x) - g(x)|\, dx$,
то в ней оно будет неполно.
\end{problem}

\begin{problem}[\S 2.6]
Найти пополнение метрического пространства, состоящего из непрерывных финитных на числовой оси функций с метрикой $\rho(x,y) = \max_t |x(t) - y(t)|$.
\end{problem}

\begin{problem}[\S 2.7]
Существует ли числовая функция, непрерывная в рациональных и разрывная в иррациональных точках
отрезка $[0,1]$?
\end{problem}

\begin{problem}[\S 4.1]
Докажите, что нормированное пространство полно $\Leftrightarrow$ в нём
всякий абсолютно сходящихся ряд сходится.
\end{problem}

\begin{problem}[\S 4.7]
Пусть $B_1$ и $B_2$ --- шары в нормированном пространстве с радиусами $r_1$ и $r_2$.
Докажите, что если $B_1 \subset B_2$, то $r_1 \le r_2$.
\end{problem}

\begin{problem}[\S 4.8]
Пусть $B_1 \supset B_2 \supset \ldots$ --- последовательность вложенных замкнутых шаров в банаховом пространстве. Докажите, что $\bigcap_{k=1}^\infty B_k \ne \emptyset$.
\end{problem}

\begin{problem}[Т-1.10]
Пусть $1\le p < q \le +\infty$. Докажите, что $(\ell_p, \|\cdot\|_q)$
является неполным, представьте его в виде счётного объединения нигде не плотных подмножеств, и постройте его пополнение, состоящее из числовых последовательностей.
\end{problem}

\begin{problem}[Т-1.11]
Пусть $1\le p < q \le +\infty$. Докажите, что линейное нормированное пространство $(L_q[0,1],\|\cdot\|_p)$ является неполным, а его пополнением является пространство $(L_p[0,1], \|\cdot\|_p)$.
\end{problem}

\begin{problem}[Т-1.12]
Пусть $F$ --- линейное пространство непрерывных и ограниченных на $\R$ вещественных функций, норма в котором имеет вид
\begin{equation*}
\|f\| = \int_{-\infty}^{+\infty} \frac{|f(x)|}{1+x^2} \, dx, \qquad f\in F.
\end{equation*}
Исследуйте пространство $(F,\|\cdot\|)$ на полноту и сепарабельность. Если
пространство $(F, \|\cdot\|)$ окажется неполным, то постройте его пополнение
(состоящее из вещественных функций). % ред.!
\end{problem}

\section{Второе задание (срок сдачи 10-14 декабря)}

\subsection{Компактные и вполне ограниченные множества в метрических пространствах}

\begin{problem}[\S 3.4]
Пусть $X$ --- метрическое пространство, обладающее тем свойством, что любая непрерывная функция
на нём ограничена. Докажите, что $X$ --- компакт.
\end{problem}

\begin{problem}[\S 3.8]
Докажите, что компактное метрическое пространство сепарабельно.
\end{problem}

\begin{problem}[\S 3.10]
Докажите, что компакт нельзя изометрично отобразить на своё собственное подмножество.
\end{problem}

\begin{problem}[\S 3.11]
Докажите, что множество $M$ в $\ell_2$ компактно $\Leftrightarrow$ оно замкнуто,
ограничено и 
\begin{equation*}
\forall \eps>0 \quad \exists n \quad \forall x\in M \quad \sum_{k=n}^\infty |x(k)|^2 < \eps.
\end{equation*}
(Здесь $x=(x(1), x(2), \ldots)$.) % ред.
\end{problem}

\begin{problem}[\S 3.12]
Пусть $E$ --- компактное метрическое пространство с метрикой $\rho(\cdot, \cdot)$.
Пусть $f\colon E\to E$, причём $\rho(f(x),f(y))<\rho(x,y)$ для всех $x\ne y$.
Докажите, что $f$ имеет неподвижную точку.
Верно ли, что неподвижная точка единственна?
Верно ли, что $f$ --- сжимающее отображение?
\end{problem}

\begin{problem}[Т-2.1]
Пусть множество
\begin{equation*}
S=\left\{f\in C^1[0,1] \;\middle|\; \|f\|_{C^1} \overset{def}{=} \|f\|_C + \|f'\|_C = 1\right\}.
\end{equation*}
\begin{enumerate}
\item[а)] Исследуйте $S$ в пространстве $(C[0,1], \|\cdot\|_C)$ на вполне ограниченность и замкнутость.
\item[б)] Исследуйте замыкание $S$ в пространстве $(C^1[0,1], \|\cdot\|_C)$ на вполне ограниченность и полноту.
\item[в)] Исследуйте $S$ в пространстве $(C^1[0,1], \|\cdot\|_{C^1})$ на вполне ограниченность и замкнутость.
\end{enumerate}
\end{problem}

\begin{problem}[Т-2.2]
Исследуйте ограниченность, вполне ограниченность и замкнутость множества
\begin{equation*}
\begin{aligned}
1) \; & S_1 = \left\{f\in L_1[0,1] \;\middle|\; 0 \le f(x) \le \frac{1}{\sqrt{x}} \;\; \text{п.в. } x\in (0,1)\right\} \\
2) \; & S_2 = \left\{f\in C^1[0,1] \;\middle|\; f(0)=0, \;\; |f'(x)|\le \frac{1}{\sqrt{x}}\;\; \forall x\in (0,1)\right\}
\end{aligned}
\end{equation*}
в пространстве $L_1[0,1]$.
\end{problem}

\begin{problem}[Т-2.3]
Исследуйте ограниченность, вполне ограниченность и замкнутость множеств
\begin{equation*}
\begin{aligned}
1) \; & S_1 = \left\{x \in c_0 \;\middle|\; \exists f\in L_1[0,1], \;\; \|f\|_1 \le 1, \;\; \forall k\in \N 
\;\; x(k)=\int_{2^{-k}}^{2^{1-k}} f(t)\, dt \right\} \\
2) \; & S_2 = \left\{x \in c_0 \;\middle|\; \exists f\in L_2[0,1], \;\; \|f\|_2 \le 1, \;\; \forall k\in \N 
\;\; x(k)=\int_{2^{-k}}^{2^{1-k}} f(t)\, dt \right\}
\end{aligned}
\end{equation*}
в пространстве $c_0$.
\end{problem}

\begin{problem}[Т-2.4]
Исследуйте ограниченность, вполне ограниченность и замкнутость множеств
\begin{equation*}
\begin{aligned}
1) \; & S_1 = \left\{f \in C[0,1] \;\middle|\; \exists g\in C[0,1], \;\; \|g\|_1 \le 1, \;\; \forall x\in [0,1] \;\; f(x) = \int_0^x g(t)\, dt \right\} \\
2) \; & S_2 = \left\{f \in C[0,1] \;\middle|\; \exists g\in C[0,1], \;\; \|g\|_2 \le 1, \;\; \forall x\in [0,1] \;\; f(x) = \int_0^x g(t)\, dt \right\}
\end{aligned}
\end{equation*}
в пространстве $C[0,1]$.
\end{problem}

\subsection{Линейные нормированные, банаховы и гильбертовы пространства}

\begin{problem}[\S 4.2]
Докажите, что две нормы, определённые на одном и том же линейном пространстве,
эквивалентны тогда и только тогда, когда из сходимости последовательности по одной из норм
следует её сходимость по другой норме.
\end{problem}

\begin{problem}[\S 4.3]
В пространстве $C[a,b]$ рассматривается множество $M$, состоящее из многочленов $p(x)$
степени $\le 10$, удовлетворяющих условию $\int_a^b |p(x)|\, dx \le 10$.
Компактно ли множество $M$?
\end{problem}

\begin{problem}[\S 5.1]
Докажите, что норма пространства $C[a,b]$ не может порождаться 
никаким скалярным произведением.
\end{problem}

\begin{problem}[\S 5.3]
Приведите пример последовательности вложенных ограниченных замкнутых множеств из $\ell_2$,
имеющих пустое пересечение.
\end{problem}

\begin{problem}[\S 5.4]
Пусть $H$ --- сепарабельное гильбертово пространство, $\{e_k\}_{k=1}^\infty$
--- ортонормированный базис в $H$, $\{g_k\}_{k=1}^\infty$ --- ортонормированная система в $H$,
причём $\sum_{k=1}^\infty \|e_k - g_k\|^2 < \infty$.
Докажите, что $\{g_k\}_{k=1}^\infty$ является ортонормированным базисом в $H$.
\end{problem}

\begin{problem}[Т-2.5]
Докажите, что в бесконечномерном банаховом пространстве не существует счётного базиса Гамеля.
Приведите пример линейного пространства со счётным базисом Гамеля.
\end{problem}

\begin{problem}[Т-2.6]
В линейном пространстве $\ell_1$ постройте две нормы такие, что некоторая последовательность элементов
$\ell_1$ является сходящейся по каждой из них к разным элементам $\ell_1$.
Покажите, что пару норм с таким свойством можно построить на любом линейном бесконечномерном пространстве.
\end{problem}

\subsection{Линейные ограниченные операторы, обратный оператор}

\begin{problem}[\S 6.2]
Оператор в $\R^n_p$ задан матрицей $A$. Выразите норму оператора
через коэффициенты матрицы в случаях $p=1$, $p=2$, $p=\infty$.
Докажите неравенство $\|A\|_2^2 \le \|A\|_1 \|A\|_\infty$.
\end{problem}

\begin{problem}[\S 6.3]
Пусть $E_1$ и $E_2$ --- нормированные пространства, $A\colon E_1 \to E_2$ --- линейный оператор.
Верно ли, что $A$ непрерывен, если
а)~$\dim E_1 < \infty$; б)~$\dim E_1 = \infty$.
\end{problem}

\begin{problem}[\S 6.6]
Докажите, что следующие операторы являются линейными ограниченными и найдите их нормы:
\begin{enumerate}
\item $A\colon C[0,1] \to C[0,1]$, \quad $(Ax)(t) = \int_0^t x(s)\, ds$;
\item $A\colon C[-1,1] \to C[-1,1]$, \quad $(Ax)(t) = \int_{-1}^t x(s)\, ds - \int_{0}^1 sx(s)\, ds$;
\item $A\colon L_1[0,1] \to L_1[0,1]$, \quad $(Ax)(t) = x(\sqrt{t})$;
\item $A\colon L_2[0,1] \to L_2[0,1]$, \quad $(Ax)(t) = t\int_0^1 x(s)\, ds$.
\end{enumerate}
\end{problem}

\begin{problem}[\S 6.13]
Пусть $X,Y$ --- банаховы пространства, $A_n \in \mathcal L(X,Y)$, $n\in \N$;
$A_n x \to Ax$ на любом элементе $x\in X$. Докажите, что если $x_n \to x$, то $A_n x_n \to A x$.
\end{problem}

\begin{problem}[\S 6.22]
Докажите, что последовательность операторов $\{A_n\}$, $A_n \in \mathcal L(C[0,1])$,
$(A_n f)(x) = f(x^{1+\frac{1}{n}})$ поточечно сходится к $I$.
Верно ли, что $A_n$ сходится к $I$ по операторной норме?
\end{problem}

\begin{problem}[\S 7.5]
Рассмотрим оператор $A\colon C[0,1] \to C[0,1]$
\begin{equation*}
(Ax)(t) = \int_0^t x(s) \, ds.
\end{equation*}
Что представляет собой множество значений оператора $A$?
Существует ли оператор $A^{-1}$, определённый на множестве значений,
и ограничен ли он?
\end{problem}

\begin{problem}[\S 7.6]
Рассмотрим оператор $A\colon C[0,1] \to C[0,1]$
\begin{equation*}
(Ax)(t) = \int_0^t x(s) \, ds + x(t)
\end{equation*}
Докажите, что $A$ имеет ограниченный обратный на всем $C[0,1]$ и найдите $A^{-1}$.
\end{problem}

\begin{problem}[\S 7.8]
Докажите, что оператор $A\colon C[0,1]\to C[0,1]$
\begin{equation*}
(Ax)(t) = x(t) + \int_0^1 e^{s+t} x(s) \, ds
\end{equation*}
непрерывно обратим и найдите $A^{-1}$.
\end{problem}

\begin{problem}[Т-2.7]
Пусть $X$ и $Y$ --- линейные нормированные пространства такие, что
$Y$ нетривиально, а $X$ бесконечномерно. Докажите, что существует
неограниченный линейный оператор $A\colon X\to Y$.
% ред.: $X$ нетривиально, а $Y$ бесконечномерно -> $Y$ нетривиально, а $X$ бесконечномерно
\end{problem}

\begin{problem}[Т-2.8]
Пусть $X$ --- бесконечномерное банахово пространство относительно нормы $\|\cdot\|$.
\begin{enumerate}
\item[а)] Докажите, что существует такой неограниченный линейный оператор $A\colon X\to X$,
что его ядро тривиально, а множество значений равно $X$, и обратный оператор неограничен.
\item[б)] Докажите, что $\|x\|_A = \|Ax\|$, $x\in X$, является нормой в $X$, относительно
которой оно также является банаховым пространством.
\item[в)] Можно ли показать, что одна из двух рассматриваемых нормированных топологий в $X$
слабее другой?
\end{enumerate}
\end{problem}

\subsection{Компактные операторы}


\begin{problem}[\S 12.1]
Оператор $A\colon C[0,1] \to C[0,1]$ определяется равенством
\begin{equation*}
(Ax)(t) = \int_0^1 K(t,s) x(s) \, ds + \sum_{k=1}^n \varphi_k(t) x(t_k),
\end{equation*}
где $K(t,s)$ непрерывна при $0\le s,t \le 1$, $\varphi_k\in C[0,1]$, $t_k \in [0,1]$.
Докажите, что $A$ компактен.
\end{problem}

\begin{problem}[\S 12.7]
Может ли оператор $A\colon C[0,1] \to C[0,1]$
\begin{equation*}
(Ax)(t) = \int_0^1 K(t,s) x(s) \, ds,
\end{equation*}
где $K(t,s)$ непрерывна при $0 \le s, t \le 1$, иметь ограниченный обратный?
\end{problem}

\begin{problem}[\S 12.8]
Пусть $X$ --- банахово пространство, $A \in \mathcal L(X)$ и существует
$c>0$ такое, что для любого $x\in X$ выполнено $\|Ax\|\ge c\|x\|$.
Может ли оператор $A$ быть компактным?
\end{problem}

\begin{problem}[\S 12.10]
Является ли преобразование Фурье $F\colon X\to Y$, $F[f](x) = \int_\R f(y) e^{ixy} \, dy$,
компактным оператором в случае
\begin{enumerate}
\item[а)] $X=Y=L_2(\R)$;
\item[б)] $X=L_1(\R)$, $Y=C_b(\R)$.
\end{enumerate}
\end{problem}

\begin{problem}[\S 12.11]
Пусть $E$ --- банахово, $H$ --- гильбертово пространства.
Пусть $A\in \mathcal L(E,H)$ --- компактный оператор.
Докажите, что существует последовательность $\{A_n\}\subset \mathcal L(E,H)$
такая, что $\dim \mathop{\mathrm{Im}} A_n < \infty$ для всех $n$,
а $\|A_n - A\| \to 0$ при $n\to \infty$.
\end{problem}

\end{document}